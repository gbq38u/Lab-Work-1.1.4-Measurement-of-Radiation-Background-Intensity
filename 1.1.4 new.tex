\documentclass[a4paper,10pt]{article} % Тип документа

% --- Пакеты и кодировки ---
\usepackage[left=3cm,right=3cm]{geometry}   % поля
\usepackage{float}                           % [H] для рисунков

\usepackage[utf8]{inputenc}                  % кириллица в исходнике (на новом LaTeX можно опустить)
\usepackage[T2A]{fontenc}                    % шрифтовая кодировка
\usepackage[russian]{babel}                  % локализация/переносы
\usepackage{csquotes}

% --- Графика (ВАЖНО: без dvips!) ---
\usepackage{graphicx}                        % Должен идти до \graphicspath
\graphicspath{{./}{pictures/}}               % ищем картинки и в текущей папке, и в pictures/
\DeclareGraphicsExtensions{.pdf,.png,.jpg}

% --- Математика ---
\usepackage{amsmath,amsfonts,amssymb,amsthm,mathtools}

% --- Ссылки/закладки ---
\usepackage{hyperref}                        % обычно подключают перед bookmark
\usepackage{bookmark}

% --- Границы плавающих объектов ---
\usepackage[section]{placeins}               % для \FloatBarrier

% --- Титул ---
\title{Отчёт о выполнении лабораторной работы\\
\textbf{Измерение интенсивности радиационного фона}\\
\textbf{Лабораторная работа 1.1.4}}
\author{Г.А. Кузин}
\date{ Группа Б01-501 г. Долгопрудный, ФРКТ МФТИ, 18.09.2025 \\}

\begin{document}
\maketitle
\newpage

\section*{Аннотация}
В работе исследованы распределения актов регистрации радиоактивного излучения с помощью счётчика Гейгера–Мюллера. Проведены измерения числа импульсов за фиксированные интервалы времени (10, 20, 40, 80 секунд), построены гистограммы распределения числа зарегистрированных частиц. Для каждого интервала рассчитаны среднее число регистрируемых частиц, среднеквадратичное отклонение, погрешность среднего значения. Экспериментальные результаты сопоставлены с распределениями Пуассона и Гаусса. Полученные данные подтвердили, что стандартное отклонение растёт пропорционально квадратному корню из среднего числа зарегистрированных счётов, что соответствует статистической природе радиоактивного распада.

\newpage
\section*{Введение}
При изучении природных и техногенных источников ионизирующего излучения важную роль играют методы его регистрации. Такие исследования необходимы для контроля радиационной обстановки, обеспечения безопасности на предприятиях атомной промышленности, в медицине при работе с рентгеновскими и радиоизотопными установками, а также в научных экспериментах по ядерной и космической физике.
Одним из наиболее распространённых приборов для регистрации ионизирующего излучения является счётчик Гейгера–Мюллера. Он позволяет фиксировать отдельные взаимодействия частиц или квантов гамма-излучения с веществом и тем самым изучать статистические закономерности радиоактивного распада. Анализ распределения числа счётов за фиксированные интервалы времени даёт возможность экспериментально подтвердить вероятностный характер процессов распада и сопоставить результаты с теоретическими моделями распределения Пуассона и Гаусса.
Целью данной работы являлось экспериментальное исследование статистических флуктуаций числа частиц, регистрируемых счётчиком Гейгера–Мюллера, построение распределений за различные интервалы времени и сравнение полученных данных с теоретическими предсказаниями.

\newpage
\section*{Методика эксперимента}

Для исследования статистики радиоактивного распада используется счётчик ионизированных частиц и источник излучения. Основная цель измерений — определить, как распределяется число зарегистрированных частиц за одинаковые промежутки времени, и сравнить эксперимент с теоретическими законами распределения.
В ходе эксперимента фиксируется последовательность значений $n_i$, каждое из которых соответствует числу зарегистрированных импульсов за выбранный интервал времени $t$. Повторяя измерение много раз (с числом серий $N$), формируется статистическая выборка. Дальнейшая обработка основана на анализе этой выборки.
Для количественного описания результатов вводятся следующие характеристики:
\begin{enumerate}
  \item \textbf{Среднее число зарегистрированных частиц} $\bar{n}$:
  \[
    \bar{n} = \frac{1}{N} \sum_{i=1}^{N} n_i .
  \]
  \item \textbf{Среднеквадратичное отклонение} $\sigma$:
  \[
    \sigma = \sqrt{\frac{1}{N}\sum_{i=1}^{N}(n_i - \bar{n})^2}.
  \]
  \item \textbf{Погрешность среднего} $\sigma_{\bar{n}}$:
  \[
    \sigma_{\bar{n}} = \frac{\sigma}{\sqrt{N}}.
  \]
  \item \textbf{Интенсивность счётов} $I$:
  \[
    I = \frac{\bar{n}}{t}.
  \]
\end{enumerate}

Полученное распределение числа событий сравнивается с теоретическими законами: распределением Пуассона (для малых $\bar{n}$) и гауссовым распределением (предельный случай при больших $\bar{n}$).

\newpage
\section*{Результаты и их обсуждение}
С помощью счетчика Гейгера–Мюллера производилась автоматическая регистрация числа частиц на широте города Долгопрудный.  
Перед началом работы было включено оборудование и программное обеспечение. В ходе эксперимента фиксировались распределения актов регистрации радиоактивного излучения за выбранные интервалы времени, а полученные данные сохранялись для дальнейшей обработки.  
На этапе обработки результаты были сгруппированы по интервалам времени. Для каждого интервала быд сформирован статистический ряд распределения, который использовался для построения гистограммы: по оси абсцисс откладывалось число зарегистрированных частиц $n$, по оси ординат — частота их появления.  
Для каждого интервала времени построены таблицы распределения числа зарегистрированных импульсов (см. приложение) и соответствующие гистограммы с наложенными теоретическими кривыми распределений Пуассона и Гаусса. [2]
\begin{figure}[H]
\centering
\includegraphics[width=0.8\textwidth]{hist10.png}
% Ensure hist10.png exists in ./ or ./pictures/
\caption{Гистограмма (10 с) с наложенными кривыми Пуассона и Гаусса.}
\end{figure}

\begin{figure}[H]
\centering
\includegraphics[width=0.8\textwidth]{hist20.png}
% Ensure hist20.png exists in ./ or ./pictures/
\caption{Гистограмма (20 с) с наложенными кривыми Пуассона и Гаусса.}
\end{figure}

\begin{figure}[H]
\centering
\includegraphics[width=0.8\textwidth]{hist40.png}
% Ensure hist40.png exists in ./ or ./pictures/
\caption{Гистограмма (40 с) с наложенными кривыми Пуассона и Гаусса.}
\end{figure}

\begin{figure}[H]
\centering
\includegraphics[width=0.8\textwidth]{hist80.png}
% Ensure hist80.png exists in ./ or ./pictures/
\caption{Гистограмма (80 с) с наложенными кривыми Пуассона и Гаусса.}
\end{figure}
Анализ гистограмм распределения числа зарегистрированных импульсов и сравнительных графиков по критерию согласия $\chi^2$ показывает, что для короткого интервала измерений (10 с) экспериментальные данные лучше описываются распределением Пуассона, что видно как по форме гистограммы, так и по меньшим вкладам в $\chi^2$ по сравнению с гауссовским распределением. При увеличении интервала (20 и 40 с) различие между моделями уменьшается: обе кривые удовлетворительно совпадают с экспериментом, но распределение Пуассона сохраняет небольшое преимущество. Для самого длинного интервала (80 с) распределения Пуассона и Гаусса практически неразличимы: вклады в $\chi^2$ малы для обеих моделей, что соответствует теоретическому предсказанию о сближении распределений при больших средних числах событий. Таким образом, результаты эксперимента подтверждают: при малых интервалах времени статистика регистрации блииже к распределению Пуассона чем к распределению Гаусса, а при росте интервала распределение плавно приближается к распределению Гаусса.
\FloatBarrier
\clearpage
\section*{Выводы}
В ходе эксперимента с использованием счётчика Гейгера–Мюллера были зарегистрированы частицы радиоактивного излучения, сгруппированные по временным интервалам различной длительности (10, 20, 40 и 80 секунд). Построенные гистограммы показали, что форма распределений зависит от длины интервала: при малом времени (10 с) экспериментальные данные лучше описываются распределением Пуассона, что подтверждается как визуальным совпадением гистограммы с теорией, так и анализом критерия согласия $\chi^2$. При увеличении интервала (20 и 40 с) различие между Пуассоном и нормальным распределением уменьшается, а для самого длинного интервала (80 с) оба закона практически неразличимы и дают одинаково хорошее описание экспериментальных данных. Таким образом, результаты подтверждают теоретическое предсказание: при малых интервалах времени регистрация частиц подчиняется распределению Пуассона, а при увеличении числа событий распределение плавно переходит к нормальному, при этом стандартное отклонение растёт пропорционально квадратному корню из среднего числа зарегистрированных частиц.
\newpage
\section*{Приложение}
Показано, сколько раз встречалось то или иное значение числа зарегистрированных частиц. Таблица служит основой для построения гистограммы и проверки согласия с распределениями Пуассона и Гаусса.
\begin{table}[H]
\centering
\caption{Распределение числа импульсов за интервал 10 с}
\begin{tabular}{|c|c|c|}
\hline
Число импульсов $n$ & Число случаев & Доля случаев \\
\hline
3 & 1 & 0.0025 \\
4 & 2 & 0.0050 \\
5 & 1 & 0.0025 \\
6 & 9 & 0.0225 \\
7 & 19 & 0.0475 \\
8 & 19 & 0.0475 \\
9 & 38 & 0.0950 \\
10 & 37 & 0.0925 \\
11 & 37 & 0.0925 \\
12 & 39 & 0.0975 \\
13 & 46 & 0.1150 \\
14 & 28 & 0.0700 \\
15 & 48 & 0.1200 \\
16 & 29 & 0.0725 \\
17 & 17 & 0.0425 \\
18 & 13 & 0.0325 \\
19 & 7 & 0.0175 \\
20 & 4 & 0.0100 \\
21 & 3 & 0.0075 \\
22 & 1 & 0.0025 \\
23 & 1 & 0.0025 \\
25 & 1 & 0.0025 \\
\hline
\end{tabular}
\end{table}
Распределение числа зарегистрированных частиц за интервал 20 секунд.
С увеличением интервала среднее число зарегистрированных частиц возрастает, а относительные колебания становятся меньше.(видно из построенных выше гистограм и графиков)
\begin{table}[H]
\centering
\caption{Распределение числа импульсов за интервал 20 с}
\begin{tabular}{|c|c|c|}
\hline
Число импульсов $n$ & Число случаев & Доля случаев \\
\hline
14 & 1 & 0.0050 \\
15 & 2 & 0.0100 \\
16 & 4 & 0.0200 \\
17 & 6 & 0.0300 \\
18 & 8 & 0.0400 \\
19 & 9 & 0.0450 \\
20 & 12 & 0.0600 \\
21 & 8 & 0.0400 \\
22 & 13 & 0.0650 \\
23 & 8 & 0.0400 \\
24 & 13 & 0.0650 \\
25 & 13 & 0.0650 \\
26 & 13 & 0.0650 \\
27 & 16 & 0.0800 \\
28 & 11 & 0.0550 \\
29 & 17 & 0.0850 \\
30 & 15 & 0.0750 \\
31 & 8 & 0.0400 \\
32 & 8 & 0.0400 \\
33 & 5 & 0.0250 \\
34 & 4 & 0.0200 \\
35 & 1 & 0.0050 \\
36 & 2 & 0.0100 \\
38 & 1 & 0.0050 \\
\hline
\end{tabular}
\end{table}
Распределение числа зарегистрированных частиц за интервал 40 секунд.
При большем интервале распределение становится более симметричным, что соответствует приближению к нормальному закону. (видно из построенных выше гистограм и графиков)
\begin{table}[H]
\centering
\caption{Распределение числа импульсов за интервал 40 с}
\begin{tabular}{|c|c|c|}
\hline
Число импульсов $n$ & Число случаев & Доля случаев \\
\hline
33 & 1 & 0.0100 \\
35 & 1 & 0.0100 \\
36 & 1 & 0.0100 \\
37 & 1 & 0.0100 \\
38 & 3 & 0.0300 \\
39 & 3 & 0.0300 \\
40 & 2 & 0.0200 \\
41 & 5 & 0.0500 \\
42 & 6 & 0.0600 \\
43 & 3 & 0.0300 \\
44 & 7 & 0.0700 \\
45 & 5 & 0.0500 \\
46 & 4 & 0.0400 \\
47 & 5 & 0.0500 \\
48 & 6 & 0.0600 \\
49 & 6 & 0.0600 \\
50 & 6 & 0.0600 \\
51 & 6 & 0.0600 \\
52 & 7 & 0.0700 \\
53 & 3 & 0.0300 \\
54 & 2 & 0.0200 \\
55 & 5 & 0.0500 \\
56 & 2 & 0.0200 \\
57 & 3 & 0.0300 \\
58 & 3 & 0.0300 \\
59 & 1 & 0.0100 \\
60 & 3 & 0.0300 \\
61 & 1 & 0.0100 \\
62 & 1 & 0.0100 \\
63 & 1 & 0.0100 \\
64 & 2 & 0.0200 \\
65 & 1 & 0.0100 \\
67 & 1 & 0.0100 \\
\hline
\end{tabular}
\end{table}



\begin{table}[H]
\centering
\caption{Распределение числа импульсов за интервал 80 с}
\begin{tabular}{|c|c|c|}
\hline
Число импульсов $n$ & Число случаев & Доля случаев \\
\hline
74 & 1 & 0.0200 \\
76 & 1 & 0.0200 \\
77 & 1 & 0.0200 \\
78 & 2 & 0.0400 \\
79 & 1 & 0.0200 \\
80 & 1 & 0.0200 \\
81 & 2 & 0.0400 \\
82 & 2 & 0.0400 \\
83 & 3 & 0.0600 \\
84 & 2 & 0.0400 \\
85 & 3 & 0.0600 \\
86 & 2 & 0.0400 \\
87 & 3 & 0.0600 \\
88 & 1 & 0.0200 \\
89 & 2 & 0.0400 \\
90 & 2 & 0.0400 \\
91 & 3 & 0.0600 \\
92 & 2 & 0.0400 \\
93 & 1 & 0.0200 \\
94 & 2 & 0.0400 \\
95 & 1 & 0.0200 \\
96 & 1 & 0.0200 \\
97 & 3 & 0.0600 \\
98 & 3 & 0.0600 \\
99 & 1 & 0.0200 \\
100 & 3 & 0.0600 \\
101 & 1 & 0.0200 \\
102 & 2 & 0.0400 \\
103 & 1 & 0.0200 \\
104 & 1 & 0.0200 \\
105 & 1 & 0.0200 \\
106 & 1 & 0.0200 \\
107 & 1 & 0.0200 \\
108 & 1 & 0.0200 \\
109 & 2 & 0.0400 \\
110 & 2 & 0.0400 \\
111 & 1 & 0.0200 \\
112 & 1 & 0.0200 \\
113 & 1 & 0.0200 \\
115 & 1 & 0.0200 \\
118 & 1 & 0.0200 \\
119 & 1 & 0.0200 \\
121 & 1 & 0.0200 \\
\hline
\end{tabular}
\end{table}

\begin{thebibliography}{99}
\bibitem{Gladun}
Д.\,А. Гладун, \textit{Лабораторный практикум по общей физике}. — М.: Издательство, 2012.

\bibitem{Sivuchin}
Д.\,В. Сивухин, \textit{Общий курс физики. Механика. Т.~1}. — ФИЗМАТЛИТ, 2016.
\end{thebibliography}

\end{document}
